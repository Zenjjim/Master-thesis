\subsection{Majority Vote Method}\label{subsec:vote}

The majority vote method, introduced by \cite{abdelfattah2021zero}, is an approach for ranking candidate architectures based on the pooled results of multiple zero-cost proxy metrics. This technique consolidates the rankings generated by different zero-cost proxy metrics, and the architectures are ranked according to the majority vote derived from these metrics. The algorithm of the voting method is outlined in \cref{alg:vote}. 
\clearpage

\begin{algorithm}[h!]
\caption{Voting Accuracy for Metric Combinations}\label{alg:vote}
\begin{algorithmic}[1]
\Function{vote}{$mets$, $gt$}
    \State $numpos \leftarrow 0$ \Comment{Initialize the number of positive elements}
    \For{each element $m$ in $mets$}
        \If{$m > 0$}
            \State Increment $numpos$ by 1
        \EndIf
    \EndFor
    \If{majority of elements in $mets$ are positive}
        \State $sign \leftarrow +1$
    \Else
        \State $sign \leftarrow -1$
    \EndIf
    \State \Return $sign * gt$ 
\EndFunction
\vspace{1em}
\Function{calc}{$acc$, $metrics$, $comb$}
    \State $num\_pts$ \Comment{Initialize the total number of data points }
    \State $tot \leftarrow 0$, $right \leftarrow 0$
    \For{each pair of distinct indices $i$ and $j$}
        \State $diff \leftarrow acc[i] - acc[j]$
        \If{$diff \neq 0$}
            \State $diffsyn$ \Comment{Initialize an empty list}
            \For{each metric $m$ in $comb$}
                \State $diffsyn \leftarrow metrics[m][i] - metrics[m][j]$
            \EndFor
            \State \textit{/* Check if $diffsyn$ and $diff$ have same sign */}
            \State $same\_sign \leftarrow$ \Call{vote}{$diffsyn$, $diff$}
            \If{$same\_sign > 0$}
                \State Increment $right$ 
            \EndIf
            \State Increment $tot$
        \EndIf
    \EndFor
    \State $votes \leftarrow \frac{right}{tot}$ \Comment{Calculate the voting accuracy}
    \State \Return $(comb, votes)$
\EndFunction
\end{algorithmic}
\end{algorithm}


Given that it is uncertain which combination of zero-cost proxies would yield the best results, the authors developed a function to generate all possible subsets of the 14 zero-cost proxy metrics. Subsequently, the majority vote for each subset was calculated and compared to the ground truth provided by the validation accuracy of every trained architecture in the benchmark. For example, \cref{tab:example_architectures} displays two architectures with given values for three zero-cost proxies (\gls{synflow}, \gls{SNIP} and Grad Sign) and their validation accuracy. 

\begin{table}[h]
\centering
\caption{Example of two architectures with validation accuracy and zero-cost proxy metrics.}
\begin{tabular}{ll}
\textbf{Architecture} & \textbf{Metrics}          \\ \hline
\multicolumn{1}{l|}{1} & \multicolumn{1}{l}{Validation Accuracy: 0.85} \\
\multicolumn{1}{l|}{\cellcolor{verylightgray}} & \cellcolor{verylightgray}Synflow: 0.62 \\
\multicolumn{1}{l|}{} & \multicolumn{1}{l}{SNIP: 0.75} \\
\multicolumn{1}{l|}{\cellcolor{verylightgray}} & \cellcolor{verylightgray}Grad Sign: 0.28 \\ \hline
\multicolumn{1}{l|}{2} & \multicolumn{1}{l}{Validation Accuracy: 0.89} \\
\multicolumn{1}{l|}{\cellcolor{verylightgray}} & \cellcolor{verylightgray}Synflow: 0.48 \\
\multicolumn{1}{l|}{} & \multicolumn{1}{l}{SNIP: 0.82} \\
\multicolumn{1}{l|}{\cellcolor{verylightgray}} & \cellcolor{verylightgray}Grad Sign: 0.36
\end{tabular}
\label{tab:example_architectures}
\end{table}

By examining the validation accuracy, one can deduce that Architecture 2 is superior to Architecture 1. Upon analysing the computed zero-cost proxy metrics, the following observations can be made:

\begin{itemize}
\item \textbf{Synflow}: Architecture 1 $(0.62) >$ Architecture 2 $(0.48)$
\item \textbf{SNIP}: Architecture 1 $(0.75) <$ Architecture 2 $(0.82)$
\item \textbf{GradSign}: Architecture 1 $(0.28) <$ Architecture 2 $(0.36)$
\end{itemize}

In this case, one positive difference (\gls{Synflow}) and two negative differences (\gls{SNIP} and Grad Sign). Consequently, the majority vote favours Architecture 2, consistent with the validation accuracy.