\section{Conclusion}

This thesis has explored the application of zero-cost proxies in NAS with GCN for HAR tasks. Through comprehensive analysis and experimentation, it has been demonstrated that using zero-cost proxies can enhance the efficiency and accuracy of NAS algorithms. The experiments show that the best-performing zero-cost proxies exhibit a very strong correlation of a Spearman $\rho$ of $\approx 0.8$, indicating that some of the zero-cost proxies can rank architectures. Since the benchmark consists of high-performing architectures, the results imply that the best zero-cost proxies correlate very strong with high-performing architectures. In addition, the thesis showed no improvement regarding warm-up, as no significant correlation was discovered after training the architectures compared to the initialisation. The results showed that some of the proxies did improve, but considering it requires training and the improvement was not significant, it can be inferred that further research and optimisation are necessary to enhance their performance and reliability. Finally, vote and weighted arithmetic mean were implemented to combine zero-cost proxies, and the result showed that the combination yields potential but not any substantial improvement compared to using each zero-cost proxy individually. 
 

The limitations of this study have been acknowledged, including the relatively small dataset size and the dependence on the GCN-NAS framework for NAS acceleration. These limitations emphasize the necessity for continued research and validation to ensure the generalizability and robustness of the findings. Furthermore, the significance of considering the environmental implications of this work has been stressed, as the advancement of more efficient and sustainable AI algorithms and techniques is vital for addressing the escalating concerns surrounding the carbon footprint and energy consumption in AI research.

With regards to the overall goal of the thesis, namely to \textit{improve and optimise the efficiency of neural architecture search with
graph convolutional networks for human action recognition}, the thesis has shown that there is great potential in using zero-cost proxies within a NAS algorithm. Especially as the experiments show that the best zero-cost proxies have a spearman $\rho$ of $\approx 0.8$, there is no doubt that utilising zero-cost proxies in a NAS algorithm will be far more efficient than today. \cite{abdelfattah2021zero} showed that utilising zero-cost proxies in different NAS algorithms (Random Search, Reinforcement Learning, Aging evolution and Binary Predictor) exhibits great improvement in efficiency, which backs our statement that the study's findings will have a positive impact on NAS in GCN for HAR.