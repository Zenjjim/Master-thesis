\section{Future Work}

Considering the findings and limitations of this thesis, various directions for additional future research are suggested.

\subsection{Expand dataset size}
Augmenting the number of trained architectures in the dataset would bolster the generalizability and statistical significance of the results. Furthermore, by incorporating more architectures with diverse characteristics, future research could examine more nuanced relationships between zero-cost proxies and the performance of GCN models within HAR tasks.

\subsection{Investigate other zero-cost proxy combination techniques}
Exploring alternative methods for combining zero-cost proxies, such as supervised learning models (neural networks or decision trees) or unsupervised learning approaches (clustering), could improve the efficiency and accuracy of architecture search in NAS algorithms.

\begin{comment}
\subsection{Develop new zero-cost proxy metrics}
Identifying new zero-cost proxy metrics with stronger correlations to validation accuracy could further enhance the efficacy of combined zero-cost proxy metrics in NAS for GCN.
\end{comment}

\subsection{Explore multiple GCN NAS frameworks}
Utilising various GCN NAS frameworks would allow for a more comprehensive comparative analysis, evaluation of the robustness of the acceleration technique, and investigation of different search strategies, optimisation methods, and search space configurations.

\subsection{Validate acceleration technique across different problem domains}
Testing the NAS acceleration method across diverse problem domains would ensure its adaptability to various search spaces and hyperparameter configurations and provide a comprehensive evaluation of its performance.

By following these research directions, the field of NAS can continue progressing, particularly in the context of GCN for HAR, and contribute to developing more efficient, accurate, and sustainable AI algorithms and techniques.

\subsection{Incorporate zero-cost proxies in a NAS algorithm}
Incorporating zero-cost proxies into NAS algorithms has excellent potential for improving their performance and efficiency by reducing search time with no cost of accuracy. Zero-cost proxies are good at predicting validation accuracy at fully trained, which means they can be used to estimate the performance of GCN architectures without costly training. This approach can be used to optimise different aspects of NAS algorithms, such as accuracy and efficiency.
\begin{comment}
This research has practical implications for healthcare and sports science, where accurate and efficient HAR is essential for monitoring patient health or athlete performance.     
\end{comment}
Overall findings suggest that incorporating zero-cost proxies into NAS algorithms has excellent potential for improving their performance and efficiency in various applications while reducing search time with no cost of accuracy.