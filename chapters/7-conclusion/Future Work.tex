\section{Future Work}

Considering the findings and limitations of this thesis, various directions for additional future research are suggested.

\paragraph{Expand dataset size}
Augmenting the number of trained architectures in the dataset would bolster the generalisability and statistical significance of the results. Furthermore, by incorporating more architectures with diverse characteristics, future research could examine more nuanced relationships between zero-cost proxies and the performance of \gls{GCN} models within \gls{HAR} tasks.

\paragraph{Investigate other zero-cost proxy combination techniques}
Exploring alternative methods for combining zero-cost proxies, such as supervised learning models (neural networks or decision trees) or unsupervised learning approaches (clustering), could improve the efficiency and accuracy of architecture search in \gls{NAS} algorithms.

\paragraph{Explore multiple GCN NAS frameworks}
Utilising various \gls{GCN} \gls{NAS} frameworks would allow for a more comprehensive comparative analysis, evaluation of the robustness of the acceleration technique, and investigation of different search strategies, optimisation methods, and search space configurations.

\paragraph{Validate acceleration technique across different problem domains}
Testing the \gls{NAS} acceleration method across diverse problem domains would ensure its adaptability to various search spaces and hyperparameter configurations and provide a comprehensive evaluation of its performance.

By following these research directions, the field of \gls{NAS} can continue progressing, particularly in the context of \gls{GCN} for \gls{HAR}, and contribute to developing more efficient, accurate, and sustainable AI algorithms and techniques.

\paragraph{Incorporate zero-cost proxies in a NAS algorithm}
Incorporating zero-cost proxies into \gls{NAS} algorithms has excellent potential for improving their performance and efficiency by reducing search time with no cost of accuracy. Zero-cost proxies are good at predicting validation accuracy at fully trained, which means they can be used to estimate the performance of \gls{GCN} architectures without costly training. This approach can be used to optimise different aspects of \gls{NAS} algorithms, such as accuracy and efficiency.

Overall findings suggest that incorporating zero-cost proxies into \gls{NAS} algorithms has excellent potential for improving their performance and efficiency in various applications while reducing search time with no cost of accuracy.