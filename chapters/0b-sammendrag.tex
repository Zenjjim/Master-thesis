\chapter*{Sammendrag}

I det raskt utviklende feltet Graph Convolutional Networks (\glspl{GCN}), er arkitekturvalgprosessen gjennom Neural Architecture Search (\gls{NAS}) fortsatt en avgjørende, men utfordrende oppgave. Hovedfokuset i denne avhandlingen er å undersøke anvendelsen og ytelsen til zero-cost proxies for evaluering av \glspl{GCN} innenfor oppgaver for gjenkjennelse av menneskelig aktivitet (\gls{HAR}). Basert på behovet for mer forskning i feltet, tar studien sikte på å bygge bro over dette gapet ved å evaluere forskjellige zero-cost proxies på \gls{GCN}-arkitekturer. 

Så vidt vi vet, er studien den første i litteraturen til å vise hvordan zero-cost proxies presterer på \glspl{GCN}. Forfatterne gjorde en kvantitativ studie på hvordan zero-cost proxies presterer ved å lage en omfattende benchmark av trente arkitekturer for \gls{HAR}-oppgaver. 

Gjennom en serie analyser og eksperimenter, viste studien at integrering av zero-cost proxies kan øke effektiviteten og nøyaktigheten til \gls{NAS}-algoritmer betydelig. Resultatene viser at de zero-cost proxiesene som presterer best viste en Spearman $\rho$ på omtrent $0.8$, noe som indikerer en veldig sterk korrelasjon. Imidlertid ble ingen betydelig forbedring i korrelasjon oppdaget når arkitekturene ble analysert etter å ha blitt trent i flere epoker, noe som antyder at zero-cost proxies er mest effektive ved initialiseringen av det nevrale nettverket. Forsøk på å kombinere zero-cost proxies ved hjelp av stemmegiving og vektet aritmetisk gjennomsnitt viste potensial, men ga ikke noen betydelig forbedring sammenlignet med å bruke zero-cost proxies individuelt. 