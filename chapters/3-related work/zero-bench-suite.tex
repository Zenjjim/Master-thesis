\subsection{NAS Bench Suite Zero}
\cite{krishnakumar2022bench} evaluated 13 zero-cost proxies on 28 different tasks and is thus the most extensive dataset for zero-cost proxies in the literature. This dataset may be vital in conducting faster experiments of zero-cost proxies as it offers precomputed zero-cost proxies scores on all tasks. To demonstrate the usefulness of the dataset, the authors conducted significant analyses of the different proxies, such as a bias analysis. 

The article demonstrates that a technique is available to enhance the efficiency of a zero proxy by reducing biases. In this situation, biases may include a tendency to prefer more extensive architectures or those with more convolutions. In the paper, the authors consider the biases given in \cref{tab:biases}. 

\begin{table}[ht]
\caption{List of biases}
\centering
\begin{tabular}{|l}
conv:pool\\
\cellcolor{verylightgray}cell size    \\
num. skip connections\\
\cellcolor{verylightgray}num. parameters              
\end{tabular}
\label{tab:biases}
\end{table}

Through their research, they find that although many zero-cost proxies demonstrate different forms of biases to varying degrees, it is possible to reduce these biases and thus enhance their performance.