\section{Summary and Implications}

\Cref{sec:rel_performance} presents the existing literature on performance predictors in \gls{NAS} algorithms and the variety of types available. Further, the chapter shows that prior work has been performed on zero-cost proxies, but there is a lack of research on zero-cost proxies on \glspl{GCN}. However, some work concerning \gls{NAS} for \gls{GCN} has been done, as shown in \cref{sec:nas_gcn}. Still, the lack of performance predictors in the papers is a significant limitation, leaving a critical gap in evaluating and comparing proposed architectures effectively. 

Applying zero-cost proxies in \gls{NAS} for \glspl{GCN} could potentially offer valuable insights into the performance of different network architectures without the need for expensive and time-consuming training processes.

Moreover, investigating zero-cost proxies for \glspl{GCN} could also improve the efficiency of \gls{NAS} algorithms. This would allow for a more rapid and accurate selection of optimal architectures and the discovery of novel \gls{GCN} architectures more suited to specific tasks or datasets.

The limitations and gaps highlight the necessity of this thesis. Thus, the next logical step in this area of research is to conduct empirical testing and evaluation of zero-cost proxies within the context of \glspl{GCN}.
