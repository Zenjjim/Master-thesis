\section{Implications}

Chapter 3 has highlighted the current state-of-the-art in the field. \Cref{sec:rel_performance} shows the existing literature on performance predictors in NAS algorithms and the variety of types available. Further, the chapters show that prior work has been performed on zero-cost proxies, but there is a lack of research on zero-cost proxies on GCNs. However, some work concerning NAS for GCN has been done, as shown in \cref{sec:nas_gcn}. Still, the lack of performance predictors in the papers is a significant limitation, leaving a critical gap in evaluating and comparing proposed architectures effectively. 

Applying zero-cost proxies in NAS for GCNs could potentially offer valuable insights into the performance of different network architectures without the need for expensive and time-consuming training processes.

Moreover, investigating zero-cost proxies for GCNs could also improve the efficiency and effectiveness of NAS processes. Not only would this allow for a more rapid and accurate selection of optimal architectures, but it could also lead to the discovery of novel GCN architectures that are more suited to specific tasks or datasets.

The limitations and gaps underscore the necessity of this thesis. Thus, the next logical step in this area of research is to conduct empirical testing and evaluation of zero-cost proxies within the context of GCNs.
