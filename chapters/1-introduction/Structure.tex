\section{Thesis outline}

The thesis consists of the following seven chapters:

\textbf{Chapter 1 - Introduction:} The introductory chapter presents the study's background, motivation, problem statement, scope, goal, and research questions. The main goal is to investigate the potential of using zero-cost proxies to enhance the efficiency and accuracy of \gls{NAS} algorithms for \gls{GCN} applied to \gls{HAR} tasks. 

\textbf{Chapter 2 - Theory:} Chapter 2 introduces the challenges of using machine learning with graphs and the emergence of \gls{GCN}s as a solution. It describes the core of \glspl{GCN} and the graph convolution operation and introduces \gls{AutoML} and \gls{NAS} as a subfield of \gls{AutoML} and the field of \gls{HAR}. 

\textbf{Chapter 3 - Related Work:} A comprehensive review of the literature on \gls{NAS} and \gls{GNN} \gls{NAS} is provided, encompassing zero-cost proxies and recent research on \gls{GNN} \gls{NAS}. In addition, a discussion on the gap and limitations in the literature is presented. 

\textbf{Chapter 4 - Method:} This chapter covers the use of zero-cost proxies in \gls{NAS}, including various proxies and their implementation, the use of warmup, and the combination of proxies using supervised learning and weighted arithmetic mean. It also evaluates the effectiveness of the methods used in the study.

\textbf{Chapter 5 - Results:} The results from the investigation into the use of zero-cost proxies in \gls{NAS} for \gls{GCN} are presented, including a correlation and vote section, and address the three research questions concerning the use of proxies in \gls{NAS} for \gls{GCN}.

\textbf{Chapter 6 - Discussion:} This chapter discusses the study's findings on improving and optimising the efficiency of \gls{NAS} with \gls{GCN} for \gls{HAR}. It analyses and interprets the results and their implications and acknowledges the study's limitations.

\textbf{Chapter 7 - Conclusion and Future Work:} The conclusion of the thesis summarises the findings and discusses future work in the field.