\section{Problem Statement}\label{ProblemStatement}
In recent years, NAS has emerged as a promising technique for automating designing neural networks with excellent performance on specific tasks \autocite{zoph2016neural}. Several studies have shown that NAS can effectively identify suitable architectures for GCN problems, widely used in social network analysis, bioinformatics applications and human action recognition. However, despite the recent advancements in NAS for GCN, evaluating architectures in existing studies remains a significant challenge. It is often infeasible to thoroughly train each candidate's architecture to obtain its ground truth accuracy \autocite{zoph2016neural}. This issue is particularly pressing in resource-constrained environments, where training large numbers of models is infeasible.

Moreover, a significant limitation of the existing literature on GCN-NAS is the absence of performance predictors, further complicating the optimisation process. The lack of such predictors can make it challenging to identify the most promising architectures for a given task efficiently. Evaluating every architecture can be prohibitively time-consuming and computationally expensive. Novel approaches, such as zero-cost proxies, have been proposed to predict candidate architectures' performance without requiring full training. However, there is still much work to be done in exploring the efficacy of these approaches and optimising them for GCN-NAS.