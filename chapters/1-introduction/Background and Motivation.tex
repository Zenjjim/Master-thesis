 \section{Background and Motivation}
This thesis is written as part of a project named DeepInMotion \autocite{deepinmotion}. Researchers from the \Gls{NTNU} and St. Olavs Hospital cooperate in a cross-disciplinary collaboration involving child physiotherapists, paediatricians, neonatologists, movement scientists and computer engineers. The project has developed a pipeline for detecting \Gls{CP} in infants by providing a video of the infant's movement. The pipeline employs a \Gls{CNN} to accurately extract movement from 2D images or videos \autocite{groos2021efficientpose}. A \gls{GCN} then processes the output to predict \gls{CP} in high-risk infants at three months of age \autocite{groos2022convolutional}. 

The \gls{CP}-prediction pipeline used \gls{NAS} to find a suitable architecture automatically. However, the search could be faster and more efficient while still finding optimal architectures. Performance prediction offers a way to predict an architecture's relative performance for a specific issue. Compared to training the architecture until convergence, this approach may be considerably more effective. Recent studies \autocite{abdelfattah2021zero, colin2022adeeperlook} show that zero-cost proxies yield great promise regarding using it to rank different architectures on image classification tasks. However, to our knowledge, research is yet to be done on how zero-cost proxies perform on \gls{GCN} architectures. Consequently, additional studies in this field may provide important findings which can be used to improve the \gls{CP}-prediction pipeline. Also, additional knowledge about how zero-cost proxies can be used with graph-based learning could be gained. 


