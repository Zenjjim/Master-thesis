 \section{Background and Motivation}
This thesis is written as a minor part of a project named InMotion. Researchers from the \Gls{NTNU} and St. Olavs Hospital cooperate in a cross-disciplinary collaboration involving child physiotherapists, paediatricians, neonatologists, movement scientists and computer engineers. The project has developed a pipeline for detecting \Gls{CP} in infants by providing a video of the infant's movement. The pipeline employs a \Gls{CNN} to accurately extract movement from 2D images or videos \autocite{groos2021efficientpose}. A GCN then processes the output to predict CP in high-risk infants at three months of age \autocite{groos2022convolutional}. 

The CP-prediction pipeline used NAS to automatically find a suitable architecture which yields good results in regards to the performance of the model. However, the search could be faster and more efficient while providing good results. Performance predictor is a way of estimating the relative performance of an architecture for a given problem and can be far more effective than training the architecture until convergent. There needs to be more research on using zero-cost proxies with GCN. This gap in the literature means that potential improvements in the efficiency and performance of NAS when using GCN might still need to be discovered. The CP-prediction pipeline could be enhanced by addressing this gap, and additional knowledge about how zero-cost proxies can be used with graph-based learning could be gained. This research aims to explore the effectiveness of zero-cost proxies in conjunction with GCN and uncover how they perform. 