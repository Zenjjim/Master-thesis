\section{Goal and Research Questions}\label{section:goalsandrq}

This section highlights the overall goal and the research questions of this thesis. The goal outlines what the research ultimately seeks to achieve, while the research question lays out the central issues the study will address to accomplish this goal. Together, these guide the research design, methodology, and analysis. 

\textbf{Goal} \textit{Improve and optimise the efficiency of neural architecture search with graph convolutional networks for human action recognition.} 

\gls{NAS} has been used with \gls{GCN} in different studies \autocite{zhou2019auto, groos2022toward, peng2020learning}, but finding other, more efficient methods is still possible. If one can find ways far more effective than what exists today, more architectures can be researched, which may result in detecting other well-performing architectures. Also, as training and searching for neural networks may impact the environment, effective methods will significantly reduce the carbon footprint. 

\textbf{Research question 1} \textit{How well can different zero-cost proxies rank \gls{GCN} architectures compare to their validation accuracy?}

Recent studies \autocite{abdelfattah2021zero, colin2022adeeperlook} show that zero-cost proxies yield great promise regarding using it to rank different architectures. However, to the authors´ knowledge, research is yet to be done on how zero-cost proxies perform on \gls{GCN} architectures. Evaluating the correlation between zero-cost proxies and ground truth validation accuracy can determine if they accurately indicate the ground truth.


\textbf{Research Question 2} \textit{How early can we identify the correlation between zero-cost proxies and validation accuracy during the warm-up phase of \gls{GCN} training to potentially halt the training process sooner?}


Research Question 2 aims to investigate the potential for early identification of the correlation between zero-cost proxies and validation accuracy during the warm-up phase of training for \gls{GCN}. In addition, this research question aims to determine if it is possible to score network before fully training it. 


\textbf{Research question 3} \textit{How can we effectively combine zero-cost proxies using various techniques to enhance the efficiency and accuracy of architecture search in \gls{NAS} algorithms?}

Through investigating Research Question 3, the study aims to identify effective techniques for combining zero-cost proxies to enhance the efficiency and accuracy of architecture search. By leveraging the strengths of multiple zero-cost proxies, future \gls{NAS} algorithms may become more efficient and accurate in discovering high-performing architectures. The outcomes of this research question can provide insights into how to optimise the use of zero-cost proxies in \gls{NAS} algorithms and improve the architecture search process in the future.