\section{Deep Learning} 
Deep learning, a subfield of machine learning, has gained significant attention in recent years due to its ability to learn hierarchical representations from raw data, especially in domains such as computer vision, natural language processing, and speech recognition. This contrasts conventional machine-learning techniques, which are limited in processing the same \autocite{lecun2015deep}. 

The core building blocks of deep learning are \Gls{ANN} inspired by the biological neural networks found in the human brain. \glspl{ANN} consist of interconnected layers of artificial neurons called nodes, each receiving input from previous layers, processing the information, and propagating the output to the subsequent layers. Deep learning architectures typically involve multiple layers of these interconnected nodes, hence the term "deep" \autocite{goodfellow2016deep}.

One key advantage of deep learning over traditional machine learning techniques is its ability to automatically learn and extract features from raw data without relying on manual feature engineering. This process, called representation learning, enables deep learning models to understand hierarchical representations of the input data, with each layer capturing increasingly abstract and complex features \autocite{bengio2013representation}. This capability has led to breakthrough performance improvements in various applications, including image classification, natural language understanding, and speech recognition \autocite{krizhevsky2017imagenet}.