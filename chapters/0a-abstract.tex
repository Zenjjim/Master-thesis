\chapter*{Abstract}

In the rapidly evolving field of Graph Convolutional Networks (GCNs), the architecture selection process through Neural Architecture Search (NAS) remains a crucial yet challenging task. The core focus of this thesis is to investigate the application and performance of zero-cost proxies for evaluating GCNs within the context of Human Activity Recognition (HAR) tasks as a first step towards using them in a NAS algorithm. Furthermore, given the need for more research, the study aims to bridge this gap by evaluating different zero-cost proxies on GCN architectures. 

To the best of our knowledge, the study is the first in the literature to exhibit how zero-cost proxies perform on GCNs. In addition, the authors did a quantitative study on how zero-cost proxies perform by creating an extensive benchmark of trained architectures for HAR tasks.

Through a series of analyses and experiments, the study demonstrated that integrating zero-cost proxies can significantly enhance the efficiency and accuracy of NAS algorithms. The results revealed that the top-performing zero-cost proxies displayed a Spearman $\rho$ of approximately $0.8$, indicating a very strong correlation. However, no substantial improvement in correlation was detected when analysing architectures as they were trained for several epochs, implying that the zero-cost proxies might be most efficient at the initialisation of the neural network. Attempts to combine zero-cost proxies using vote and weighted arithmetic mean showed potential but did not yield considerable improvement compared to the individual application of each zero-cost proxy. 