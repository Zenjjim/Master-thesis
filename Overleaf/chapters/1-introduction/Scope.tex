\section{Scope of the Thesis}

This thesis investigates the potential of utilising zero-cost proxies to enhance the efficiency of architecture search in \gls{NAS} algorithms, specifically targeting \gls{GCN} applied to \gls{HAR} tasks. The research questions formulated for this study are designed to explore various aspects of zero-cost proxies and their effectiveness in ranking \gls{GCN} architectures.

Nonetheless, it is crucial to outline the scope and limitations of this thesis to ensure that readers understand which topics will and will not be discussed. Although the overall goal is to improve and optimise the efficiency of \gls{NAS} with \gls{GCN} for \gls{HAR}, the primary focus of this study is to analyse and evaluate different zero-cost proxies. As a result, a full implementation of the research findings within an \gls{NAS} algorithm will not be provided.

By clarifying the scope of this thesis, the intention is to offer readers a more comprehensive understanding of the research focus and the study's limitations. While recognising that incorporating the findings into a practical \gls{NAS} algorithm is an essential and valuable subsequent step, the primary objective of this thesis is to lay the foundation for future research by exploring the potential of zero-cost proxies in the context of \gls{GCN} for \gls{HAR} tasks.