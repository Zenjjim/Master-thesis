\section{Thesis outline}

The thesis consists of the following seven chapters:

\textbf{Chapter 1 - Introduction:} The introductory chapter presents the study's background, motivation, problem statement, scope, goal, and research questions.

\textbf{Chapter 2 - Theory:} Chapter 2 introduces necessary background theory, including deep learning, \gls{NAS}, \gls{GCN} and \gls{HAR}.  

\textbf{Chapter 3 - Related Work:} A comprehensive review of the literature on \gls{NAS} and \gls{GCN}-\gls{NAS} is provided, encompassing zero-cost proxies and recent research on \gls{GCN}-\gls{NAS}. In addition, a discussion on the gap and limitations in the literature is presented. 

\textbf{Chapter 4 - Method:} This chapter covers the use of zero-cost proxies in \gls{NAS}, including various proxies and their implementation, the use of warmup, and the combination of proxies using a custom vote measure and weighted arithmetic mean. 

\textbf{Chapter 5 - Results:} The results, namely the correlation analysis, vote measure and weighted arithmetic mean, are presented in this chapter. 

\textbf{Chapter 6 - Discussion:} This chapter discusses the study's findings and analyses and interprets the results and their implications. In addition, it discusses the study's limitations as well as the environmental implications.

\textbf{Chapter 7 - Conclusion and Future Work:} The conclusion of the thesis summarises the findings and discusses future work. 