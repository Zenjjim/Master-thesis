\section{Human Action Recognition (HAR)}
\glsreset{HAR}
\gls{HAR} is a method that interprets the human body's gestures or motions via sensors, accelerometers or videos and uses these to predict human action \autocite{jobanputra2019human}. Like many other computer vision tasks, \gls{HAR} can be supervised and unsupervised, where supervised training requires a large amount of data \autocite{ann2014human}. We can classify \gls{HAR} into two problems; the localisation problem and the recognition problem. The localisation problem concerns where something is located in the video, whereas the recognition problem concerns the type of action we see \autocite{vrigkas2015review}. 

Due to problems like background clutter, partial occlusion, and changes in scale and frame resolution, capturing the specific action of a human within a video is a challenging task \autocite{vrigkas2015review}. However, modelling the human body as three-dimensional data is another approach that has emerged in recent years. The human body can be divided into connecting joints forming a three-dimensional structure.  