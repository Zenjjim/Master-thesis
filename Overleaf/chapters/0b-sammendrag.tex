\chapter*{Sammendrag}

Både Neural Architecture Search (NAS) og Graph Convolutional Networks (GCNs) er to felter innen maskinlæring som har gjennomgått en stor utvikling de seneste årene. Det å finne en GCN-arkitektur som gir gode resultater kan være svært tidskrevende og ressurskrevende. Zero-cost proxyer, som er utformet for å kun kreven en enkelt minibatch med treningsdata for å score et neural nettverk, har blitt introdusert for å gjøre denne prosessen mer effektiv. Hovedfokuset med denne avhandlingen er å undersøke bruken og ytelsen av zero-cost proxies for å evaluere GCN innenfor oppgaver relatert til gjenkjenning av menneskelig aktivitet (HAR) som et første steg mot å bruke dem i en NAS-algoritme. Basert på behovet for videre forskning i feltet, tar studien sikte på å bygge bro over dette gapet ved å evaluere forskjellige zero-cost proxies på GCN-arkitekturer. Så vidt vi vet, er studien den første til å utforske hvordan zero-cost proxies presterer på GCN. 

Gjennom en serie analyser og eksperimenter, viser studien at integrering av zero-cost proxies kan betydelig forbedre effektiviteten til NAS-algoritmer. Resultatene viser at de best presterende zero-cost proxyene viser en Spearman Rank Correlation ($\rho$) på omtrent $0.8$, noe som indikerer en sterk til veldig sterk korrelasjon. Imidlertid blir ingen betydelig forbedring i korrelasjon oppdaget når arkitekturene blir analysert etter å ha blitt trent i flere epoker, noe som antyder at zero-cost proxies er mest effektive ved initialiseringen av det nevrale nettverket. Forsøk på å kombinere zero-cost proxier ved hjelp av stemmegiving og vektet aritmetisk gjennomsnitt viser potensial, men gir ikke noen betydelig forbedring sammenlignet med å bruke zero-cost proxies individuelt. 





